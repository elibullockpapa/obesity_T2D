{\color{gray}\hrule}
\begin{center}
\section{Setting up the simulation}
\textbf{Processing SELECT and STASIS trial data, and fitting unknown parameters to our model}
\bigskip
\end{center}
{\color{gray}\hrule}

\subsection{Data Measured in Trials}
To accurately model weight loss across different BMI categories, we analyzed data from both the SELECT and STASIS trials \cite{Ryan2024}. Our goal was to determine the true treatment effect of semaglutide assuming perfect adherence.

For the SELECT trial, we incorporated three key components:
\begin{itemize}
    \item Estimated Treatment Differences (ETD) for each BMI category
    \item Placebo group weight loss (-1.5\% at 4 years)
    \item Adherence adjustment (+1.5\% based on first on-treatment analysis)
\end{itemize}

The adherence-adjusted weight loss was calculated using:
\begin{equation}
    \text{Adjusted Weight Loss} = (\text{ETD} + \text{Placebo Loss}) + 1.5\%
\end{equation}

\begin{table}[h]
\centering
\begin{tabular}{|l|c|c|c|c|}
\hline
\textbf{BMI Group} & \textbf{ETD (\%)} & \textbf{Placebo (\%)} & \textbf{Initial (\%)} & \textbf{Adjusted (\%)} \\
\hline
BMI $<$30 & -7.52 & -1.5 & -9.02 & -10.52 \\
BMI 30-35 & -8.79 & -1.5 & -10.29 & -11.79 \\
BMI 35-40 & -9.01 & -1.5 & -10.51 & -12.01 \\
BMI $\geq$40 & -9.23 & -1.5 & -10.73 & -12.23 \\
\hline
\end{tabular}
\caption{SELECT trial weight loss percentages by BMI category, adjusted for adherence}
\end{table}

For the STASIS trial, baseline measurements varied by BMI category:

\begin{table}[h]
\centering
\begin{tabular}{|l|c|c|c|c|}
\hline
\textbf{Parameter} & \textbf{BMI $<$30} & \textbf{BMI 30-35} & \textbf{BMI 35-40} & \textbf{BMI $\geq$40} \\
\hline
Glucose (mg/dl) & 96.21 & 102.73 & 112.93 & 122.40 \\
Insulin ($\mu$U/ml) & 10.43 & 12.75 & 15.19 & 15.60 \\
FFA ($\mu$mol/l) & 424.02 & 481.58 & 565.77 & 642.29 \\
Inflammation & 0.11 & 0.22 & 0.40 & 0.57 \\
\hline
\end{tabular}
\caption{STASIS trial baseline measurements by BMI category}
\end{table}

\subsection{STEP Trial Data}
We analyzed data from the STEP trial, which provided both baseline and 2-year measurements for several key parameters, as well as percentage changes for others.

\begin{table}[h]
\centering
\begin{tabular}{|l|c|c|c|}
\hline
\textbf{Parameter} & \textbf{Baseline} & \textbf{2-Year} & \textbf{Change (\%)} \\
\hline
\multicolumn{4}{|c|}{\textit{Directly Measured Values}} \\
\hline
Weight (kg) & 105.6 & 89.5 & -15.2 \\
Glucose (mg/dl) & 95.4 & 88.2 & -7.5 \\
Insulin ($\mu$U/ml) & 12.6 & 8.5 & -32.7 \\
\hline
\multicolumn{4}{|c|}{\textit{Percentage Changes Only}} \\
\hline
FFA ($\mu$mol/l) & -- & -- & +0.3 \\
Inflammation & -- & -- & -56.7 \\
\hline
\end{tabular}
\caption{STEP trial measurements showing both absolute values and percentage changes}
\end{table}

\subsection{Model Parameter Selection}
To run our simulation, we needed values for all model parameters. These came from three sources:
\begin{enumerate}
    \item Direct measurements from the STEP trial
    \item Default model values (when measurements weren't available)
    \item Fitted values (when default values proved unstable)
\end{enumerate}

\begin{table}[h]
\centering
\begin{tabular}{|l|c|c|c|l|}
\hline
\textbf{Parameter} & \textbf{Default} & \textbf{Used} & \textbf{Source} & \textbf{Notes} \\
\hline
Glucose & 94.1 & 95.4 & Measured & From STEP baseline \\
Insulin & 9.6 & 12.6 & Measured & From STEP baseline \\
Weight & 81.0 & 105.6 & Measured & From STEP baseline \\
Height & 1.8 & 1.65 & Measured & From STEP baseline \\
Age & 30 & 47.3 & Measured & From STEP baseline \\
\hline
FFA & 400 & 688.65 & Fitted & Default unstable \\
Si & 0.8 & 0.26 & Fitted & Default unstable \\
Beta & 1009 & 1009 & Default & Stable value \\
Sigma & 530 & 501.64 & Fitted & Default unstable \\
Inflammation & 0.056 & 0.45 & Fitted & Default unstable \\
\hline
\end{tabular}
\caption{Model parameter sources and values. "Fitted" values were determined by running a 2-week simulation and observing where parameters stabilized when default values proved unstable.}
\end{table}

\subsection{Parameter Stability Analysis}
For parameters where we used fitted values instead of defaults, this decision was based on observing significant instability in the first two weeks of simulation. A parameter was considered unstable if it showed rapid deviation from its initial value, suggesting the default was physiologically unlikely for the STEP trial population.

For example:
\begin{itemize}
    \item FFA increased from 400 to 688.65 $\mu$mol/l
    \item Si decreased from 0.8 to 0.26 ml/$\mu$U/day
    \item Inflammation increased from 0.056 to 0.45
\end{itemize}

These fitted values were then used as initial conditions for the full simulation, resulting in more stable trajectories that better matched the observed clinical outcomes.

\subsection{Adjusting for our Model}
To translate these findings into our simulation framework, we first converted BMI categories to target weights using our model subject's height (1.8m). We then performed a binary search to determine both the pre-treatment caloric intake needed to reach each BMI category and the treatment-phase intake required to achieve the observed weight loss.

This process yielded the following caloric requirements:

\begin{table}[!htb]
\centering
\begin{tabular}{|l|c|c|c|c|c|c|}
\hline
\textbf{BMI Group} & \textbf{Initial} & \textbf{Initial} & \textbf{Final} & \textbf{Final} & \textbf{Weight} \\
& \textbf{Weight (kg)} & \textbf{Calories} & \textbf{Weight (kg)} & \textbf{Calories} & \textbf{Change (\%)} \\
\hline
BMI $<$30 & 76.2 & 2,353 & 68.2 & 2,098 & -10.5 \\
BMI 30-35 & 88.5 & 2,730 & 78.0 & 2,401 & -11.8 \\
BMI 35-40 & 102.1 & 3,152 & 89.8 & 2,766 & -12.0 \\
BMI $\geq$40 & 114.3 & 3,530 & 100.4 & 3,091 & -12.2 \\
\hline
\end{tabular}
\caption{SELECT Trial Analysis Results (Pre-treatment: 7 years, Treatment: 4 years)}
\end{table}

When we simulate the model with these caloric requirements, we get the following weight trajectories:

\begin{figure}[!htb]
\centering
\includegraphics[width=0.8\textwidth]{images/wegovy_weights_plot.png}
\caption{Simulated weight trajectories by BMI category showing pre-treatment weight gain and subsequent Wegovy treatment response}
\label{fig:wegovy_weights}
\end{figure}

This matched the observed weight loss in the SELECT trial, indicating that our caloric requirements were accurate.

\subsection{Binary Search for Caloric Requirements}
To determine the caloric requirements for each phase of the trials, we implemented a binary search algorithm that finds the daily caloric intake needed to maintain a target weight. The algorithm:

\begin{enumerate}
    \item Sets an initial range of possible calories (1500-6000 kcal/day)
    \item Simulates weight trajectory for the midpoint caloric value
    \item Narrows the search range based on whether the final weight is above or below target
    \item Continues until the difference between final and target weight is within 0.1 kg
\end{enumerate}

This process yielded the following caloric requirements for the SELECT trial:

\begin{table}[h]
\centering
\begin{tabular}{|l|c|c|c|c|}
\hline
\textbf{BMI Group} & \textbf{Initial} & \textbf{Initial} & \textbf{Final} & \textbf{Final} \\
& \textbf{Weight (kg)} & \textbf{Calories} & \textbf{Weight (kg)} & \textbf{Calories} \\
\hline
BMI $<$30 & 76.2 & 2,353 & 68.2 & 2,098 \\
BMI 30-35 & 88.5 & 2,730 & 78.0 & 2,401 \\
BMI 35-40 & 102.1 & 3,152 & 89.8 & 2,766 \\
BMI $\geq$40 & 114.3 & 3,530 & 100.4 & 3,091 \\
\hline
\end{tabular}
\caption{Caloric requirements determined through binary search for SELECT trial phases}
\end{table}

And for the STEP trial:

\begin{table}[h]
\centering
\begin{tabular}{|l|c|c|c|c|}
\hline
\textbf{Initial BMI} & \textbf{Initial} & \textbf{Initial} & \textbf{Final} & \textbf{Final} \\
& \textbf{Weight (kg)} & \textbf{Calories} & \textbf{Weight (kg)} & \textbf{Calories} \\
\hline
38.6 & 105.1 & 3,245 & 88.2 & 2,634 \\
\hline
\end{tabular}
\caption{Caloric requirements determined through binary search for STEP trial phases}
\end{table}

\subsection{Model Parameter Adjustments}
For the STASIS trial, we determined model parameters through a two-step process:

1. First, we calculated the target BMI categories and corresponding weights using our model subject's height (1.8m)

2. Then, we fitted the model parameters to reach equilibrium values after a 5-year pre-treatment period with appropriate caloric intake

The resulting fitted parameters were:

\begin{table}[h]
\centering
\begin{tabular}{|l|c|c|c|c|c|}
\hline
\textbf{BMI Category} & \textbf{Parameter} & \textbf{Default} & \textbf{Fitted} & \textbf{Units} \\
\hline
\multirow{5}{*}{BMI $<$30} & Glucose & 94.1 & 96.21 & mg/dl \\
& Insulin & 9.6 & 10.43 & $\mu$U/ml \\
& FFA & 400 & 424.02 & $\mu$mol/l \\
& Si & 0.8 & 0.70 & ml/$\mu$U/day \\
& Inflammation & 0.056 & 0.11 & dimensionless \\
\hline
\multirow{5}{*}{BMI 30-35} & Glucose & 94.1 & 102.73 & mg/dl \\
& Insulin & 9.6 & 12.75 & $\mu$U/ml \\
& FFA & 400 & 481.58 & $\mu$mol/l \\
& Si & 0.8 & 0.50 & ml/$\mu$U/day \\
& Inflammation & 0.056 & 0.22 & dimensionless \\
\hline
\multirow{5}{*}{BMI 35-40} & Glucose & 94.1 & 112.93 & mg/dl \\
& Insulin & 9.6 & 15.19 & $\mu$U/ml \\
& FFA & 400 & 565.77 & $\mu$mol/l \\
& Si & 0.8 & 0.33 & ml/$\mu$U/day \\
& Inflammation & 0.056 & 0.40 & dimensionless \\
\hline
\multirow{5}{*}{BMI $\geq$40} & Glucose & 94.1 & 122.40 & mg/dl \\
& Insulin & 9.6 & 15.60 & $\mu$U/ml \\
& FFA & 400 & 642.29 & $\mu$mol/l \\
& Si & 0.8 & 0.26 & ml/$\mu$U/day \\
& Inflammation & 0.056 & 0.57 & dimensionless \\
\hline
\end{tabular}
\caption{Comparison of default model values and fitted parameters for STASIS trial simulation. The fitted values represent equilibrium states after 5 years of weight gain to reach target BMI categories.}
\end{table}

These fitted values were chosen to:
\begin{itemize}
    \item Achieve stable equilibrium at target BMI categories after 5 years
    \item Maintain physiologically plausible relationships between parameters
    \item Produce realistic responses to caloric changes during the treatment phase
\end{itemize}

It's important to note that these are not direct measurements from the STASIS trial, but rather model parameters that produce the observed BMI distributions and treatment responses when simulated.

\subsection{Creating the Ramp-Up Function}
To accurately model the gradual onset of Wegovy's appetite-suppressing effects, we implemented a ramp-up function that simulates the typical clinical titration schedule. The function gradually increases the medication's effect over time, which better reflects real-world patient experiences and helps avoid sudden caloric restrictions.

\begin{equation}
    \text{Ramp Factor} = \min\left(\frac{t - t_{\text{start}}}{t_{\text{ramp}}}, 1\right)
\end{equation}

where:
\begin{itemize}
    \item $t_{\text{start}}$ = 1825 days (5-year pre-treatment period)
    \item $t_{\text{ramp}}$ = 60 days (2-month ramp-up duration)
\end{itemize}

The caloric adjustment is then applied using:
\begin{equation}
    \text{Caloric Reduction} = \text{Target Reduction} \times \text{Ramp Factor}
\end{equation}

This gradual approach ensures that:
\begin{itemize}
    \item The treatment effect increases linearly over the first year
    \item The full effect is achieved only after complete titration
    \item The simulation better matches clinical observations of weight loss patterns
\end{itemize}


