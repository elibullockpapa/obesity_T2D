{\color{gray}\hrule}
\begin{center}
\section{Conclusions}
\bigskip
\end{center}
{\color{gray}\hrule}
\vspace{0.5cm}
\begin{multicols}{2}
This study aimed to validate our computational model's ability to simulate the metabolic effects of significant weight loss by comparing it against clinical data from the SELECT and STEP trials. Our findings reveal both the strengths and limitations of the current model, while highlighting important areas for future development.

\subsection{Key Findings}
Our model successfully captured several important aspects of obesity and T2D treatment:
\begin{itemize}
    \item Accurate prediction of weight loss trajectories across different BMI categories
    \item Realistic simulation of inflammation reduction during treatment
    \item Plausible representation of the complex interplay between metabolic parameters
\end{itemize}

However, significant discrepancies emerged when comparing detailed biomarker predictions to clinical measurements:
\begin{itemize}
    \item Consistent overestimation of glucose and insulin responses
    \item Overly pessimistic predictions for severe obesity cases
    \item Failure to capture the full therapeutic benefits observed in clinical trials
\end{itemize}

\subsection{Model Limitations}
The primary limitation of our current model is its reliance on caloric restriction as the sole mechanism of weight loss. This simplification fails to account for the multiple pathways through which GLP-1 agonists improve metabolic health:
\begin{itemize}
    \item Direct effects on insulin sensitivity
    \item Beta cell preservation and function enhancement
    \item Inflammation reduction beyond weight loss effects
    \item Changes in lipid metabolism and substrate utilization
\end{itemize}

\subsection{Future Directions}
To improve the model's clinical relevance, several enhancements should be considered:
\begin{itemize}
    \item Integration of GLP-1 specific mechanisms beyond appetite suppression
    \item Refinement of acute metabolic response parameters
    \item Addition of tissue-specific insulin sensitivity measures
    \item Incorporation of metabolic flexibility indicators
\end{itemize}

\subsection{Clinical Implications}
Despite its limitations, this model provides valuable insights into the complex pathways linking obesity and T2D. Understanding these connections is crucial for:
\begin{itemize}
    \item Optimizing treatment strategies
    \item Identifying early intervention opportunities
    \item Predicting individual treatment responses
    \item Developing more effective therapeutic approaches
\end{itemize}

In conclusion, while our model successfully captures many aspects of obesity-induced metabolic dysfunction, significant opportunities exist for improvement. Future iterations should focus on incorporating the multiple mechanisms of action of modern anti-obesity medications and not just caloric restriction. This would likely lead to more accurate predictions of treatment outcomes across the full spectrum of obesity and T2D presentations.
\end{multicols}