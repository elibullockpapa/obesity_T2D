\begin{multicols}{2}
\tableofcontents
\section{Introduction}
Obesity has become a pressing global health challenge, with alarming increases in both adult and childhood populations. In the United States alone, adult obesity rates have risen dramatically from 30.5\% in 1999-2000 to 41.9\% by 2020, affecting over 100 million adults. Even more concerning is the doubling of severe obesity rates from 4.7\% to 9.2\% during the same period. This epidemic extends to the younger generation, with approximately 20\% of children and adolescents now classified as obese\cite{CDC2024a,CDC2024b}.

This widespread metabolic disorder is a significant risk factor for Type II Diabetes (T2D), driven by a complex interplay of insulin resistance, beta-cell dysfunction, and systemic inflammation. These metabolic disturbances result in chronic hyperglycemia, which is associated with severe complications, including cardiovascular disease and neuropathy.

GLP-1 receptor agonists, particularly Semaglutide, have emerged as revolutionary treatments in obesity and T2D management. These medications have demonstrated unprecedented efficacy in clinical trials, enabling significant and sustained weight loss while simultaneously improving glycemic control. The SELECT and STEP trials have validated their effectiveness, with long-term studies showing sustained benefits over two to five years\cite{Garvey2022,Ryan2024}. The complete mechanism of action extends beyond the well-understood effects of appetite suppression and delayed gastric emptying.

\subsection{Modeling Objectives}
Building upon the computational model developed by Yildirim et al.\cite{Yildirim2023}, our study aims to validate and extend their framework to capture the effects of modern GLP-1 agonist treatments. Their original model focused exclusively on simulating the effects of caloric intake modifications, providing an ideal foundation to examine whether observed clinical outcomes can be explained through caloric restriction alone.

This validation process serves two crucial purposes. First, it tests the robustness of Yildirim's model by establishing whether caloric reduction alone can reproduce the metabolic variables associated with GLP-1 agonist treatment. Second, it provides a framework for identifying additional metabolic benefits specific to GLP-1 agonists that cannot be explained by reduced caloric intake alone. By identifying discrepancies between model predictions and clinical outcomes, we can highlight areas requiring refinement in future iterations of the model.
\end{multicols}